\begin{poglavje}
\noindent\bfseries Energetsko učinkovit sistem za detekcijo slonov s pomočjo strojnega učenja
\end{poglavje}

\bigskip
\bigskip
\bigskip
\bigskip
\bigskip
\textbf{Ključne besede:} strojno učenje, mikrokrmilnik, sklepanje na napravi, termalna kamera, sistem z majhno porabo, 
\bigskip
\textbf{UKD:} XXXXX

\bigskip
\bigskip
\bigskip
\bigskip
\textbf{Povzetek}

\lipsum[1-5]
\newpage

\begin{poglavje}
\noindent\bfseries Energy efficient system for detection of elephants with Machine Learning
\end{poglavje}

\bigskip
\bigskip
\bigskip
\bigskip
\bigskip
\textbf{Key words:} machine learning, microcontroller, on-device inference, thermal camera, low-power system

\bigskip
\textbf{UKD:} XXXXX

\bigskip
\bigskip
\bigskip
\bigskip
\textbf{Abstract}

Human-Elephant Conflict is a major environmental and animal conservative problem, according to WILDLABS, an average of 400 people and 100 elephants are killed every year in India alone because of it. 
Early warning systems replace the role of human watchers and warn local communities of nearby, potentially life threatening, elephants, thus minimising the Human-Elephant Conflicts.

In this Master's thesis we present the structure of an early warning system, which consists of several deployed embedded systems equipped with thermal cameras and a single gateway; the main focus of the thesis was the design and implementation of the former.
To detect presence of elephants, we designed and trained several accurate image classification models, capable of classifying thermal images.
We optimised said models for on-device performance and compared them in terms of accuracy, execution speed and size.
While writing firmware we ported a part of TensorFlow library and created our own build system, suitable for the platform of our choice. 
We also implemented reporting of inference results over LoRaWAN network and described possible server-size setup.
We finally constructed fully functional embedded system from various development and evaluation boards, and evaluated its performance in terms of inference speed and power consumption.
We show that embedded systems with Machine Learning capabilities are a viable solution to many real life problems.
\newpage
