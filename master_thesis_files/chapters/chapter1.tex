\section{ Motivation}

As a result of increasing human population and habitat loss, human-wildlife conflicts have become increasingly common in last several decades\cite{philip-wildlife}.
According to organization The World Wide Fund for Nature (WWF) human-wildlife conflicts are defined as: "any interaction between humans and wildlife that results in negative impacts on human social, economic or cultural life, on the conservation of wildlife populations, or on the environment"\cite{conflict-manual}.
These conflicts range from mostly harmless, non-violent contacts, such as sightings of wildlife animals in urban areas, to destruction of crops and infrastructure, killings of livestock, and in worst cases, losses of human lives.
In more severe cases these conflicts end in defensive or retaliatory killings of wildlife animals which can drive already endangered species to extinction.

Polar bears, tigers and elephants are generally considered to be most problematic \cite{philip-wildlife}.
In Arctic, as a consequence of reduction of their natural habitat, polar bears are drawn to human settlements, food dumps, while searching for food\cite{wildlabs-polarbears}.
Unexpected encounters can turn deadly for for both sides.
As wide ranging animals, tigers need large areas where they can roam, hunt and breed\cite{wildlabs-tigers}.
When their natural prey population is depleted they often turn their attention to poorly protected livestock. 
Their attacks often have economic, social and psychological consequences.
According to WILDLABS, tigers killed 101 people, between years 2013 and 2016,in India alone\cite{wildlabs-tigers}.

As herbivores, elephants might be seen as less problematic when compared to polar bears or tigers, but this assumption could not be further from the truth.
Although exact numbers vary between sources, casualties from human-elephant conflicts are much higher compared to conflicts involving polar bears or tigers.
According to WILDLABS, an average of 400 people and 100 elephants are killed every year in India\cite{wildlabs-elephants}. 
Leading cause of death of elephants is electrocution (by electric fences, unprotected power lines), followed by train accidents, poaching and poisoning\cite{cause-of-death}.
One of human-elephant conflict hotspots is in Sonitpur District in Assam province, India. 
In 5300 km\textsuperscript{2} large area around 200,000 people and 200 elephants share the same space\cite{wildlabs-elephants}.
Elephants often venture into paddy fields which represent livelihood for local communities.
A single elephant can quickly trample fields of rice crops in few hours, causing big financial problems to already impoverished farmers\cite{wildlabs-elephants}.

Several measures have been taken to minimize human-elephant conflicts:
\begin{itemize}
    \item Electrical fences, some solar powered, some directly connected to local power lines, latter are deadly for elephants
    \item Watch towers
    \item Trenches
    \item Chilli-based deterrents
    \item Regular patrols
    \item Usage of trained captive elephants to drive wild elephants into the forest
    \item Camera traps with motion sensors
\end{itemize}

Although above mentioned measures function to some degree, they are not effective enough, since they are unreliable or come into effect when the damage has already been done\cite{wildlabs}. 

\section{ Early detection system}

One important component of minimizing human-elephant conflicts is a reliable early detection system. 
System able of detecting presence of nearby elephants would warn nearby communities and give them enough time to prepare and respond nonviolently.
Same kind of system would also provide information about common elephants paths, thus giving farmers knowledge how to better construct and place their fences, barricades, to minimize human-elephant conflict.
