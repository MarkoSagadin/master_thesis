In this chapter we will describe the design of neural network that will be able to process thermal images and decide what object they contain.
Workflow that we will follow will largely be a combination of workflows presented in Figures \ref{ml_workflow} and \ref{micro_workflow}.

We will first set concrete objectives, which will dictate what exactly we want to accomplish, while keeping in consideration various constraints.
We will then explore the dataset provided by Arribada Initiative, analyze different class representations and decide if it is appropriate for accomplishing objectives that we set earlier.

After dataset exploration we had to setup our development environment.
As software for creating and training ML models consists of many components and dependencies it is important to start with something that we know it works.
We will describe our AWS server setup, how we moved image data on it and why we decided to use Python Notebooks for algorithm development.

Image preparation was important step before model creation, as wanted to keep track of various metadata labels that were assigned to each image.
This involved parsing a large excel table and connecting images with correct labels.
Image preprocessing such as mean normalization and augmentation was then done.
We then designed and trained few CNN models with varying complexities and observed how well they behave on thermal image dataset.
We then optimized models with TFLite tools and compared accuracies and model sizes.
Models were then converted into a C char array format, ready to be tested on a actual microcontroller.

We will finish this chapter by describing the same workflow, but by using tools that Edge Impulse provides to the point where we will have our model ready to run on a microcontroller.
Using Edge Impulse will generally require less knowledge, less time and will lead to better results when compared to our setup.


\section{ Objectives}

The accuracy of our early detection system should be equal or similar to the one of human observer, no matter if we are operating in the daytime or nighttime.
Although our system will be placed on the paths that are regularly traversed by elephants, this does not mean that they will be only possible object on a image taken by a thermal camera.
Humans and various livestock, such as goats and cows, could also be photographed.
This means that we want to avoid reporting false positives, which means that our system should not incorrectly label a human or a livestock animal as an elephant.
At the same time we want avoid false negatives, where an elephant could pass by our system undetected.
These kind of mistakes could undermine the community's confidence of our early detection system and defeat our purpose.
This means that besides elephant detection, we should also focus on correctly labeling humans and livestock, while also providing a nature/random class for all other unexpected objects or simply images of nature.

It would be beneficial if thermal camera can take several pictures of the same object, thus increasing the confidence of computed label of the object.
However this is constrained by the image processing time and camera's field of view.
Thermal camera FLIR Lepton has a horizontal field of view of 57 degrees.
The closer elephant passes by the thermal camera the quicker he will traverse the camera's field of view, thus giving the camera less time for capture.
This problem can be solved by optimizing the execution time of the ML model or by placing the early detection system on far enough position from expected elephant's path.
As latter option might not be always possible, we should strive to keep the whole image processing time as short as possible.

Finally, as our neural network will be deployed on a microcontroller and not on a computer or a server we have to keep it simple and small.
Extra model complexity that might bring us few percents of accuracy will not matter much, if our model would be too large to fit on a microcontroller or too slow to run.


To summarize:
\begin{itemize}
    \item We will create an image classification ML model that will be capable of processing a thermal image and sorting it into one of possible 4 categories: elephant, human, cows and nature/random.
    \item Total image processing time should be as short as possible, we should try to keep it under 2 seconds.
    \item Model should be small enough to fit on a microcontroller of our choice, while still leaving some space for application code. Microcontroller of our choice (STM32F767) has 2 MB of flash memory so model size should be smaller than that.
\end{itemize}


\section{ Exploring the dataset}
Show images, 
how and by who were they made, 
what was their original purpose
show ratios of different meta data
what is missing from dataset
Describe how you made missing pictures
make a picture of a setup

\section{ Development enviroment }
Linux server on AWS with Tensorflow preinstalled
Python Notebooks
How and why did you chose this setup
\section{ Image preparation}

\section{ Model creation and training (Rethink title)}
train few different models
with different accuracies

\section{ Optimising models}
Comparison between different sizes and accuracies

\section{ Edge Impulse (SHOULD this be here like that, how should i go about this)}
