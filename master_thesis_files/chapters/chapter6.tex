In this Master's thesis we presented a solution, an early warning system, for minimising Human-Elephant Conflicts.
In the beginning we presented a Machine Learning approach for solving classification of thermal images and we outlined its strengths compared to classical problem solving.
We described in depth the knowledge required for understanding our work from various sub parts of Machine Learning, to wireless IoT technologies and thermal cameras.

We presented several Machine Learning workflows that we followed during the design and planning of our own Neural Network.
We analysed the thermal image dataset and recognized what kind of data it lacked. 
We compensated for missing data by gathering more of them with our own image capturing setup and by resampling them.
We proposed a basic CNN structure that we used later in the model training phase.

As we wanted to execute Machine Learning algorithms on our selected STM32 microcontroller, we had to find a way to make this possible.
We ported a TensorFlow Lite for a Microcontrollers library to the libopencm3 platform and in the process created an open-source project MicroML.
In the process we familiarised ourselves with the specifics of cross compiling and build systems.

To implement our early warning system on the hardware we decided to use a two microcontroller setup, with an STM32 as a main inferencing processor and nRF52 as a low-power system controller.
To communicate with the FLIR Lepton camera we wrote the driver code, and we also wrote the LoRaWAN communication module.

We ran random search algorithms to find the proper hyperparameters for our proposed CNN model.
We did an in depth comparison of several trained models and compared then to models trained with the help of commercial software.

We saw that we could train CNN models with an accuracy as high as 98.04 \% and we saw that models trained with the Transfer Learning technique reached 98.7 \% without much additional work.
We ran each trained model on the STM32 and measured its inference time.
From the results we saw that reaching an inference time below 200 \si{\milli\second} is not hard if special care is devoted to the low number of parameters and correct microcontroller optimisations.

We also evaluated the performance of the embedded system from the battery lifetime perspective and we saw promising results.
Assuming that our system would have to process 600 events per day and that it has 6 battery cells, we estimated the lifetime of the system to be around 10 months.

This thesis shows that the field of Machine Learning on embedded devices has reached a point were it is viable enough to use it in real life applications.
Running inference directly on the embedded device can provide instant feedback and can extend the device's lifetime, as data do not need to be sent to the server for processing.
Many use cases from the Animal Conservation field would benefit from such technology, as there is a need for the embedded devices that require minimal manual care and can provide big value for their cost.


\section{ Future work}

Our research into elephant detection with the aid of Machine Learning models yielded promising results.
There are several aspects of it that can be improved.

In terms of model performance, we can always improve it by gathering more relevant training data.
The dataset that we used contained several thousand images of elephants, but only a couple of thousand images of humans and only a few hundred images of cows.
In order to train a more robust and reliable model, we should gather more thermal images of humans and livestock, especially goats.

It would be interesting to explore further models trained with the Transfer Learning technique.
We saw that Transfer Learning models reached higher accuracies with shorter inferencing times compared to other models.
We expect that running a random hyperparameter search with a smaller version of the pre-trained MobileNetV2 model could produce optimal results.

In terms of the system performance, testing our early warning system in the field would give us key insights into what could be improved.
With a device deployed in a zoo, we could monitor its performance and see which conditions degrade its performance.
We could add an SD card to the system, and save every taken image and the result of its inference.

By observing the performance of the model in the field we would see if extremely low inference times are really needed.
It might be feasible to run CNN models on slower, low-power, Cortex-M4 microcontrollers.
Although we are expecting longer inference times, we would benefit from a simpler system design and a lower overall price of the embedded system.

In terms of battery life performance, creating a custom printed circuit board specially for our application would give us more control over the current consumption of the systems. Reaching lower low-power state current consumption should be easier.
At the same time, we should research how to optimise the detection sequence. 
Decrease in either current consumption or the duration of detection sequence would provide us with huge improvements in battery life.


\section{ Final words}

Machine Learning on the embedded devices is opening doors to various, wonderful applications that were not possible a few years ago.
We can see that there is already a demand for intelligence for devices on the "edge" and we expect it to increase as the field matures.
It is quite possible that future embedded engineers would require some amount of Machine Learning knowledge to stay competitive and interesting to the market.

As Pete Warden said: "The Future of ML is Tiny and Bright".
