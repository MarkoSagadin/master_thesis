


\section{Future work}

Our research into elephant detection with the aid of Machine Learning models yielded promising results, however, some future work is needed.

In terms of model performance, we can always improve it by gathering more relevant training data.
Dataset that we used contained several thousands images of elephants, but only couple of thousands of humans and few hundreds of cows.
In order to train a more robust and reliable model we should gather more thermal images of humans and livestock.

It would be interesting to further explore models trained with Transfer Learning technique.
We saw that Transfer Learning models reached higher accuracies with shorter inferencing times compared to other models.
We expect that running a random hyperparameter search with the smaller version of pre-trained MobileNetV2 model could produce optimal results.

In terms of the system performance, testing our early warning system in field would give us key insights what could be improved.
With a device deployed in a zoo, we could monitor its performance and see which conditions degrade its performance.
We could add a SD card to the system and save every taken image, and result of its inference.

By observing performance of the model in the field we would see if extremely low inference times are really needed.
It might be feasible to run CNN models on slower, low-power, Cortex-M4 microcontrollers.
Although we are expecting longer inference times, we would benefit from a simpler system design and a lower overall price of the embedded system.

TODO You are probably not finished with this section, you need feedback.

\section{Final words}

Machine Learning on embedded devices is a field in rapid development.






Main three sections: summary, results commentary, and looking forward (future work)

, you are summarizing the paper for a reader who had read the introduction and the body of the report already, and should already have a strong sense of key concepts. Your conclusion, then, is for a more informed reader and should look quite different than the introduction.



Sklep je zadnje poglavje zaključnega dela. V njem podamo objektivno oceno rezultatov in
jih povežemo s problemom, zastavljenim v uvodu. Če se nam zdi ustrezneje, lahko opis
rezultatov in diskusijo podamo v sklepu namesto v glavnem delu. Nakažemo morebitne
težave in opažanja, ki so se nam pojavila med delom, ter podamo napotke za nadaljnje
delo.
