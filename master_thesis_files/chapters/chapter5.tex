\section{ Model comparisons}\label{model_comparisons}

As mentioned in section \ref{cnn_ref} we used Keras Tuner model to find hyperparameters that would yield the highest accuracy.
Instead of hard-coding hyperparameters when building a model with Keras API, we defined a search space of possible values with \verb|HyperParameter| class and used that as a hyperparameter.

We then passed the created model to a \verb|RandomSearch| class, with few other parameters such as batch size, number of epochs and maximum number of trials.
We then started hyperparameter search, which means that Keras Tuner was randomly picking a set of hyperparameters and training a model with them. 
This process was repeated for a trial number of times.
Accuracy and hyperparameters that were used while training every module were saved to a log file for later analysis.

After training a number of different models we handpicked a few of them and compared them.
Comparison of equivalent model trained in Edge Impulse studio was also done.


\subsection{ Hyperparameter search space and results analysis}

General structure of CNN model was already described in section \ref{cnn_ref} and in Figure \ref{model_code}.
We decided to search for following hyperparameters: number of filters in all three convolutional layers (can be different for each layer), filter size in all three convolutional layers (same for all layers), size of dense layer, dropout rate and learning rate.
Possible values of hyperparameters (also known as hyperparameter search space) are specified in table \ref{hyper_table1}.

\begin{table}
    \centering
    \begin{tabular}{@{} *5l @{}}    \toprule
        \textbf{Hyperparameter} & \textbf{Set of values}\\\midrule
        \verb|FilterNum1|       & From 16 to 80, with a step of 8\\ 
        \verb|FilterNum2|       & From 16 to 80, with a step of 8\\ 
        \verb|FilterNum3|       & From 16 to 80, with a step of 8\\
        \verb|FilterSize|       & 3 x 3 or 3 x 4\\
        \verb|DenseSize|        & From 16 to 96, with a step of 8\\
        \verb|DropoutRate|      & From 0.2 to 0.5, with a step of 0.05\\
        \verb|LearningRate|     & 0.0001 or 0.0003\\\toprule
        \textbf{Random search}  & \textbf{value}\\
        \textbf{variable}       & \\\midrule
        \verb|EPOCHS|           & 25\\
        \verb|BATCH_SIZE|       & 100\\
        \verb|MAX_TRIALS|       & 300\\\bottomrule
    \end{tabular}
    \caption{ First hyperparameter search space}
    \label{hyper_table1}
\end{table}

Search space of \verb|filter_numX|, \verb|dense_size| and \verb|dropout_rate| hyperparameters was chosen based on initial training tests and various models that were trained on similar data.
Value of \verb|filter_size| is usually 3 x 3, however all example ML projects were training on image date of same dimensions.
We wanted to test how would a filter with same ratio of dimensions as image data (3 x 4 and 60 x 80 respectively) perform.
Hyperparameter \verb|learning_rate| was chosen heuristically, we saw that higher values, such as 0.001 or 0.003, would leave model's accuracy stuck at suboptimal optima, from where it could not be improve anymore.

We also had to set 3 variables that directly affected how long will random search last.
From initial tests we saw that models usually reached maximum possible accuracy around \nth{20} epoch, to give some headroom we set the number of epochs to 25.
We kept batch size relatively small, at 100, which meant that weights would get updated regularly.
Hyperparameter \verb|MAX_TRIALS| had the biggest impact on the training time, we set it to 300.

Training lasted for about 12 hours. 
After it was done we compiled a list of all 300 trained models and their different hyperparameter values, number of parameters and accuracies, part of it can be seen in Table \ref{hyper_results1}.

\begin{table}
    \centering
    \rowcolors{2}{gray!25}{white}
    \begin{tabular}{llllllllrl}
        \textbf{Hyperparameter} & \rot{FilterNum1} & \rot{FilterNum2} & \rot{FilterNum3} & \rot{DenseSize} & \rot{DropoutRate}  &\rot{FilterSize} & \rot{LearningRate} & \rot{Number of parameters} & \rot{Accuracy[\%]}  \\\toprule
        \textbf{Model ID} &&&&&&&&\\\toprule
        0 & 72 & 80 & 64 & 72 & 0.4  & 3x4 & 0.0003 & 1.514,400 & 98.35\\
        1 & 32 & 40 & 72 & 56 & 0.35 & 3x4 & 0.0001 & 1.260,332 & 98.31\\
        2 & 40 & 48 & 32 & 64 & 0.35 & 3x4 & 0.0001 &   656,797 & 98.31\\
        3 & 56 & 16 & 48 & 72 & 0.4  & 3x4 & 0.0001 & 1,057,924 & 98.28\\
        4 & 80 & 64 & 40 & 96 & 0.45 & 3x4 & 0.0003 & 1,245,788 & 98.28\\
        5 & 64 & 24 & 72 & 88 & 0.45 & 3x3 & 0.0001 & 1,931,356 & 98.28\\
        6 & 64 & 56 & 40 & 80 & 0.35 & 3x3 & 0.0001 & 1,013,556 & 98.24\\
        7 & 16 & 40 & 64 & 88 & 0.35 & 3x3 & 0.0003 & 1,719,108 & 98.24\\
        8 & 48 & 64 & 32 & 64 & 0.35 & 3x4 & 0.0003 &   676,884 & 98.24\\
        9 & 72 & 48 & 56 & 80 & 0.3  & 3x4 & 0.0001 & 1,419,172 & 98.24\\\midrule
       91 & 72 & 48 & 56 & 40 & 0.35 & 3x3 & 0.0003 &   728,324 & 98.00\\
       92 & 48 & 24 & 64 & 64 & 0.35 & 3x4 & 0.0001 & 1,262,092 & 98.00\\
       93 & 24 & 48 & 32 & 72 & 0.4  & 3x3 & 0.0001 &   716,076 & 98.00\\
       94 & 32 & 48 & 72 & 32 & 0.25 & 3x3 & 0.0001 &   736,732 & 98.00\\
       95 & 64 & 24 & 64 & 48 & 0.3  & 3x4 & 0.0001 &   959,628 & 98.00\\
       96 & 16 & 32 & 72 & 80 & 0.25 & 3x4 & 0.0001 & 1,762,508 & 98.00\\
       97 & 72 & 56 & 40 & 56 & 0.45 & 3x4 & 0.0003 &   748,580 & 98.00\\
       98 & 32 & 24 & 24 & 48 & 0.35 & 3x3 & 0.0001 &   358,308 & 98.00\\
       99 & 48 & 16 & 40 & 40 & 0.45 & 3x3 & 0.0003 &   493,412 & 98.00\\
      100 & 24 & 72 & 64 & 40 & 0.45 & 3x3 & 0.0003 &   844,684 & 98.00\\\midrule
      191 & 64 & 56 & 16 & 52 & 0.4  & 3x3 & 0.0001 &   386,996 & 97.76\\
      192 & 48 & 40 & 24 & 24 & 0.4  & 3x4 & 0.0001 &   208,172 & 97.73\\
      193 & 56 & 64 & 72 & 24 & 0.25 & 3x4 & 0.0003 &   617,692 & 97.73\\
      194 & 48 & 72 & 48 & 32 & 0.25 & 3x4 & 0.0003 &   544,652 & 97.73\\
      195 & 72 & 56 & 24 & 56 & 0.25 & 3x4 & 0.0003 &   469,012 & 97.73\\
      196 & 72 & 48 & 72 & 40 & 0.3  & 3x3 & 0.0003 &   927,252 & 97.73\\
      197 & 80 & 16 & 32 & 80 & 0.25 & 3x3 & 0.0001 &   785,380 & 97.73\\
      198 & 56 & 24 & 16 & 88 & 0.25 & 3x3 & 0.0001 &   438,996 & 97.73\\
      199 & 56 & 24 & 16 & 88 & 0.25 & 3x3 & 0.0001 &   438,996 & 97.73\\\midrule
      295 & 48 & 32 & 64 & 16 & 0.5  & 3x4 & 0.0001 &   351,012 & 95.87\\
      296 & 40 & 24 & 56 & 24 & 0.5  & 3x4 & 0.0001 &   431,572 & 95.77\\
      297 & 56 & 16 & 80 & 16 & 0.2  & 3x4 & 0.0001 &   411,020 & 95.63\\
      298 & 24 & 16 & 48 & 24 & 0.5  & 3x4 & 0.0001 &   359,924 & 94.46\\
      299 & 40 & 48 & 56 & 16 & 0.35 & 3x3 & 0.0003 &   310,860 & 82.86\\\bottomrule
    \end{tabular}
    \caption{ Partial results of first random search of hyperparameters}
    \label{hyper_results1}
\end{table}

After analyzing results we came to several conclusions:

\begin{enumerate}
    \item We saw that almost all trained models, except of the last one, achieved accuracy above 90 \%. This proved that the general architecture of the model was appropriate for the problem.
    \item We could not see any visible correlation between a specific choice of a certain hyperparameter and accuracy.This shows that selection of hyperparameters is really a non-heuristic task.
    \item Filter of size 3 x 4 did not perform significantly better compared to one with size 3 x 3. 
    \item There is a weak correlation between number of parameters (model's complexity) and accuracy. Although eight of top ten models have more then 1 million parameters, models with IDs 2 and 8 have almost half of the parameters, but still perform well.
    \item First 200 models cover a accuracy range of 0.62 \%. However we can there models of very different size, for example model with ID 192 has more than 8 times less of parameters than model than model with ID 96, altough the difference in accuracy (0.27 \%) is neglible.
\end{enumerate}

As we realized that we do not need complex models to achieve high accuracy on our training data, we decided to run the random search of hyperparameters again.
We decided to lower the maximum and minimum numbers of filters and size of dense layer.
We also decreased the step from 8 to 2.
We decided to lower the bottom boundry of \verb|DropoutRate| from 0.2 to 0.0, which means that some models will not be using dropoout layer at all.
We expect that training without dropout layer will produce suboptimal results, however we would like to test this.
Redefined search space for second radnom search can be seen in Table \ref{hyper_table2}
We increased the number of \verb|MAX_TRIALS| from 300 to 500, as we are expecting that more models will end up underfitting and also because there will be more possible options because of smaller step size.

\begin{table}
    \centering
    \begin{tabular}{@{} *5l @{}}    \toprule
        \textbf{Hyperparameter} & \textbf{Set of values}\\\midrule
        \verb|FilterNum1|       & From 4 to 48, with a step of 2\\ 
        \verb|FilterNum2|       & From 4 to 48, with a step of 2\\ 
        \verb|FilterNum3|       & From 4 to 48, with a step of 2\\
        \verb|FilterSize|       & 3 x 3 or 3 x 4\\
        \verb|DenseSize|        & From 4 to 48, with a step of 2\\
        \verb|DropoutRate|      & From 0.0 to 0.5, with a step of 0.05\\
        \verb|LearningRate|     & 0.0001 or 0.0003\\\toprule
        \textbf{Random search}  & \textbf{value}\\
        \textbf{variable}       & \\\midrule
        \verb|EPOCHS|           & 25\\
        \verb|BATCH_SIZE|       & 100\\
        \verb|MAX_TRIALS|       & 500\\\bottomrule
    \end{tabular}
    \caption{ Second hyperparameter search space}
    \label{hyper_table2}
\end{table}

By providing accuracy metric, Keras tuner automatically sorts trained models in descending accuracy.
This is in many use cases sufficient, however in our case, we were also interested in best performing models that had small size. 

- [ ] Do model comparisons, analyze them and present them.
      This means that you need to describe the setup, 
      what you want to get from this subchapter, 
      hyperparameter search space, 
      list of models, 

\subsection{ Comparison of selected, re-trained models}
      how you will select them, 
      an image of history (accu and loss), 
    confusion matrix,
    precision and recall
\subsection{ Comparison with Edge Impulse model}
Describe how did you build Edge Impulse model, maybe mention transfer learning (might be extra work, you did not write about theory)




\section{ On device performance testing}
Describe setup
      how are you timing, 
\subsection{ Comparison of different optimization options}
\subsection{ Comparison of performance of selected models}
    Transfer learning could be interesting here, bigger number of parameters and takes less time.






\section{ Power profiling of an embedded early warning system}
- [ ] Power consumption test of whole setup, 
      PIR wakes up wisent, 
      wisent turns on stm32f7 and flir, 
      which makes a picture, does inference, reports result and wisent sends the result. 
      Otii image of consumption with marked sections.(shouldnt be hard)

\subsection{ Battery life estimations}
Based on numbers and different scenarios estimate how long would this last with different batteries.
